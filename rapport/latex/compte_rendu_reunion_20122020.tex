% \documentclass[12pt]{article}
% \usepackage[frenchb]{babel}
% \usepackage{natbib}
% \usepackage[utf8]{inputenc}
% \usepackage[T1]{fontenc}
% \usepackage{tikz}
% \usepackage{amsmath}
% \usepackage{graphics}
% \usepackage{graphicx}
% \usepackage{url}
% \usepackage{psfrag}
% \usepackage{fancyhdr}
% \usepackage{vmargin}
% \usepackage[backend=biber]{biblatex}
% \usepackage{csquotes}
% \usepackage[hidelinks]{hyperref}
% \usepackage{enumitem}

% \pagestyle{fancy}


% \begin{document}
\subsubsection*{\large{Réunion d'équipe du 20 décembre 2020}}
    \addcontentsline{toc}{subsubsection}{Réunion d'équipe du 20 décembre 2020}
\begin{center}
\begin{tabular}{| l | l || c | c |}
    \hline
    Membres présents & Membres absents & Durée & Lieu \\
    \hline
    Mohamed-Omar CHIDA & & & \\ Mathis DUMAS & & 3h & Discord \\ Chaima TOUNSI OMEZZINE & & & \\ Céline ZHANG & & & \\
    \hline
\end{tabular}
\end{center}

\subsubsection*{Ordre du jour}
\begin{enumerate}
    \item Avancement de chacun sur le projet (fin de la deuxième partie et troisième partie test, début quatrième partie)
    \item Résolution des difficultés rencontrées
    \item Optimisation des solutions proposées
    \item Vérification simpliste des tests
    \item Planification du prochain \textsl{sprint}
\end{enumerate}

\subsubsection*{Avancement de chacun sur le projet}
Nous avons discuté de l'avancement de chacun sur nos tâches après la session d'examens.
\paragraph*{Omar CHIDA} a implémenté la fonction calculant la distance de Levenshtein \texttt{lev} et a proposé une version récursive en programmation dynamique \texttt{lev\_dp}, et a commencé à implémenter la fonction qui donne l'alignement global optimal de deux séquences.

\paragraph*{Mathis DUMAS} a rédigé un début de l'introduction et de l'état de l'art (les parties concernant la distance de Levenshtein et l'algorithme de Needleman-Wunsch) et a ajouté des tests pour la fonctions \texttt{proportions}.

\paragraph*{Chaima TOUNSI OMEZZINE} s'est documentée et a réfléchi à l'implémentation de l'algorithme de Needleman-Wunsch avec Céline ZHANG, en bordant toutes les manières pour le \textsl{filling} (en utilisant une matrice de similarité ou une liste de coût de transformation) et pour le \textsl{traceback} (en comporant ou bien en utilisant une matrice qui stock les parcours).

\paragraph*{Céline ZHANG} s'est documentée et a réfléchi à l'implémentation de l'algorithme de Needleman-Wunsch avec Chaima TOUNSI OMEZZINE, en bordant toutes les manières pour le \textsl{filling} (en utilisant une matrice de similarité ou une liste de coût de transformation) et pour le \textsl{traceback} (en comporant ou bien en utilisant une matrice qui stock les parcours).

\subsubsection*{Résolution des difficultés rencontrées}
Lors de l'implémentation de l'algorithme de Needleman, il y a eu des doutes sur les situations dans lesquelles plusieurs maxima était possible, Omar CHIDA a choisi de favoriser le \textsl{match} puis le \textsl{left gap}.

\subsubsection*{Optimisation des solutions proposées}
Pour l'algorithme de Needleman-Wunsch, Céline ZHANG et Chaima TOUNSI OMEZZINE ont conseillé Omar CHIDA de sauvegarder dans une matrice parallèle le parcours menant à chaque case, ceci évite de refaire les tests pour le traceback.

% ajouter un graphe de la matrice

\subsubsection*{Vérification simpliste des tests}
Les tests de fonctions Levenshtein ont été vérifiés. Céline ZHANG avait ajouté des tests pour la fonction \texttt{taille\_ensemble} qui ont été vérifiés.

\subsubsection*{Planification du prochain \textsl{sprint}}
\paragraph*{Omar CHIDA} doit continuer la fonction \texttt{needleman} et la terminer au mieux avec les tests.

\paragraph*{Mathis DUMAS} doit continuer la rédaction de l'introduction et l'état de l'art pour le rapport ainsi que les graphes.

\paragraph*{Chaima TOUNSI OMEZZINE} doit appliquer les fonctions calculant la distance de Levenshtein sur des paires de séquences de codons.

\paragraph*{Céline ZHANG} doit faire la mise en page du rapport et organiser le rapport en \LaTeX \ sur \textsf{Overleaf}, et commencer à rédiger les réponses à la première et deuxième parties.

\paragraph{\emph{TO-DO LIST}}
\begin{itemize}
    \item Finir d'implémenter l'algorithme de Needleman-Wunsch
    \item Appliquer les fonctions de distance de Levenshtein sur les séquences de codons
    \item Continuer l'introduction et l'état de l'art
    \item Faire la mise en page et organiser le rapport
\end{itemize}

\emph{Prochaine réunion : 23/12/2020}\\

% \end{document}