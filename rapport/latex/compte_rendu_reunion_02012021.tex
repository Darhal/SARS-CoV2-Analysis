% \documentclass[12pt]{article}
% \usepackage[frenchb]{babel}
% \usepackage{natbib}
% \usepackage[utf8]{inputenc}
% \usepackage[T1]{fontenc}
% \usepackage{tikz}
% \usepackage{amsmath}
% \usepackage{graphics}
% \usepackage{graphicx}
% \usepackage{url}
% \usepackage{psfrag}
% \usepackage{fancyhdr}
% \usepackage{vmargin}
% \usepackage[backend=biber]{biblatex}
% \usepackage{csquotes}
% \usepackage[hidelinks]{hyperref}
% \usepackage{enumitem}

% \pagestyle{fancy}


% \begin{document}
\subsubsection*{\large{Réunion d'équipe du 2 Janvier 2021}}
    \addcontentsline{toc}{subsubsection}{Réunion d'équipe du 2 Janvier 2021}
\begin{center}
\begin{tabular}{| l | l || c | c |}
    \hline
    Membres présents & Membres absents & Durée & Lieu \\
    \hline
    Mohamed-Omar CHIDA & & & \\ Mathis DUMAS & & 3h & Discord \\ Chaima TOUNSI OMEZZINE & & & \\ Céline ZHANG & & & \\
    \hline
\end{tabular}
\end{center}

\subsubsection*{Ordre du jour}
\begin{enumerate}
    \item Avancement de chacun sur le projet (Fin de la quatrième partie, finalisation rapport, réflexion sur la présentation)
    \item Résolution des difficultés rencontrées
    \item Optimisation des solutions proposées
    \item Vérification simpliste des tests
    \item Planification du prochain \textsl{sprint}
\end{enumerate}

\subsubsection*{Avancement de chacun sur le projet}
\paragraph*{Omar CHIDA} a rédigé l'abstract, a écrit les tests de performances, a fait les tests de performances pour les \texttt{lev} et les \texttt{needleman}.

\paragraph*{Mathis DUMAS} a écrit l'implémentation des fonctions concernant la distance de Levenshtein, a fini l'état de l'art concernant l'algorithme de Needleman-Wunsch.

\paragraph*{Chaima TOUNSI OMEZZINE} a fait les tests de performances pour les fonctions \texttt{codons}, et a écrit la partie tests et performances pour celles-ci et les tests de Levenshtein.

\paragraph*{Céline ZHANG} a réalisé les tests de performances pour les fonctions statistiques et a écrit la partie tests et performances de celles-ci.

\subsubsection*{Résolution des difficultés rencontrées}
En raison de la complexité exponentielle des fonctions \texttt{lev\_rec} et \texttt{needleman\_all}, nous avons dû ajuster les tests de performances pour celles-ci. Et eventuellement envisager une solution \textsl{multi-threading}.

\subsubsection*{Optimisation des solutions proposées}
Un script écrit en \textsf{C} a été réalisé pour optimiser les mesures de performances, par l'exécution simultanée des fonctions. L'exécution du traçage de \texttt{needleman} a été proposée en réponse à la question~8.

\subsubsection*{Vérification simpliste des tests}
Les tests des fonctions \texttt{codons} et \texttt{needleman} ont été revues.

\subsubsection*{Planification du prochain \textsl{sprint}}
\paragraph*{Omar CHIDA} doit réalisé le traçage animé pour \texttt{needleman}, doit réaliser les mesures de performances en \textsl{multi-thread}, et doit compléter la partie tests et performances.

\paragraph*{Mathis DUMAS} doit finir les graphiques statistiques pour le rapport et doit commencer les graphes pour la présentation.

\paragraph*{Chaima TOUNSI OMEZZINE} doit ajouter des précisions pour expliquer le tableau de bord Trello et doit relire le rapport (introduction, implémentation stats, levenshtein, tests et performances stats, etc.).

\paragraph*{Céline ZHANG} doit rédiger le résumé en français du rapport, faire à nouveau l'introduction en ajoutant le cahier des charges, réaliser le bilan, et doit relire les parties du rapport (état de l'art codons, implémentation \texttt{codons}, \texttt{needleman}, tests et performances).

\paragraph{\emph{TO-DO LIST}}
\begin{itemize}
    \item Faire une exécution animée du \textsl{trace back} de \texttt{needleman}
    \item Réaliser les mesures de performances avec le \textsl{multi-threading}
    \item Finir les graphiques pour les stats et faire les graphiques pour la présentation
    \item Ajouter des précisions pour l'explication du tableau de bord Trello
    \item Corriger l'introduction, et ajouter un cahier de charges
    \item Écrire le bilan et le résumé
    \item Relire les parties du rapport
\end{itemize}

\emph{Prochaine réunion : 04/01/2021}\\

% \end{document}