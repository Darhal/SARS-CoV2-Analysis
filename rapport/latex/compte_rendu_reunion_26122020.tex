% \documentclass[12pt]{article}
% \usepackage[frenchb]{babel}
% \usepackage{natbib}
% \usepackage[utf8]{inputenc}
% \usepackage[T1]{fontenc}
% \usepackage{tikz}
% \usepackage{amsmath}
% \usepackage{graphics}
% \usepackage{graphicx}
% \usepackage{url}
% \usepackage{psfrag}
% \usepackage{fancyhdr}
% \usepackage{vmargin}
% \usepackage[backend=biber]{biblatex}
% \usepackage{csquotes}
% \usepackage[hidelinks]{hyperref}
% \usepackage{enumitem}

% \pagestyle{fancy}


% \begin{document}
\subsubsection*{\large{Réunion d'équipe du 26 décembre 2020}}
    \addcontentsline{toc}{subsubsection}{Réunion d'équipe du 26 décembre 2020}
\begin{center}
\begin{tabular}{| l | l || c | c |}
    \hline
    Membres présents & Membres absents & Durée & Lieu \\
    \hline
    Mohamed-Omar CHIDA & & & \\ Mathis DUMAS & & 3h & Discord \\ Chaima TOUNSI OMEZZINE & & & \\ Céline ZHANG & & & \\
    \hline
\end{tabular}
\end{center}

\subsubsection*{Ordre du jour}
\begin{enumerate}
    \item Avancement de chacun sur le projet (troisième partie conclusion, quatrième partie test)
    \item Résolution des difficultés rencontrées
    \item Optimisation des solutions proposées
    \item Vérification simpliste des tests
    \item Planification du prochain \textsl{sprint}
\end{enumerate}

\subsubsection*{Avancement de chacun sur le projet}
\paragraph*{Omar CHIDA} a fini d'écrire la fonction \texttt{needleman} et a fait des tests simples, a eu des problèmes de tests avec \texttt{Biopython} qui demande un paramètre \textsf{gap} en plus et ne renvoie pas forcément les mêmes solutions.

\paragraph*{Mathis DUMAS} a écrit l'introduction du rapport, a fait quelques graphes pour la première partie.

\paragraph*{Chaima TOUNSI OMEZZINE} a corrigé la fonction \texttt{codons}, a rajouté d'autres versions de celle-ci pour l'analyse statistique des séquences de codons et a écrit l'état de l'art pour la partie codons.

\paragraph*{Céline ZHANG} a appliqué les fonctions statistiques sur les échantillons des mois de janvier à avril, a test \texttt{needleman}, a trouvé un problème sur le nombre de solution et a fini la rédaction de l'état de l'art pour la partie analyse statistique du génome.

\subsubsection*{Résolution des difficultés rencontrées}
Le problème de la fonction \texttt{needleman} ne provient pas vraiment de la fonction, l'affichage des solutions partielles, et de plus \texttt{Biopython} n'utilise pas toujours le même algorithme pour les alignements, il choisit automatiquement le meilleur algorithme pour la situation. La fonction \texttt{codons} a eu un problème d'application a été résolu.


\subsubsection*{Optimisation des solutions proposées}
La fonction \texttt{codons} a été améliorée, les soucis de lecture de triplet indéterminé sont ignorés.

\subsubsection*{Vérification simpliste des tests}
Tous les tests ajoutés sont passés.

\subsubsection*{Planification du prochain \textsl{sprint}}
\paragraph*{Omar CHIDA} doit commencer la performance et doit régler le soucis de test sur \texttt{needleman} avec \texttt{Bioppython} et faire l'application pour la question 8.

\paragraph*{Mathis DUMAS} doit continuer la rédaction du rapport sur la distance de Levenshtein et éventuellement sur l'algorithme de Needleman-Wunsch.

\paragraph*{Chaima TOUNSI OMEZZINE} doit finir d'écrire l'état de l'art pour la partie concernant les codons et l'implémentation de l'algorithme de celle-ci.

\paragraph*{Céline ZHANG} doit écrire l'implémentation de l'algorithme pour la partie sur les fonctions d'analyse statistique et éventuellement continuer l'écriture de la gestion de projet.

\paragraph{\emph{TO-DO LIST}}
\begin{itemize}
    \item Faire l'application décrite par la question 8 pour \texttt{needleman}
    \item Continuer la rédaction des partie Distance de Levenshtein et Algorithme de Needleman-Wunsch
    \item Finir l'état de l'art sur la partie codons
    \item Écrire l'implémentation de l'algorithme pour l'analyse statistique et les codons
\end{itemize}

\emph{Prochaine réunion : 29/12/2020}\\

% \end{document}