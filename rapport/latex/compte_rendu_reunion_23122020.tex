% \documentclass[12pt]{article}
% \usepackage[frenchb]{babel}
% \usepackage{natbib}
% \usepackage[utf8]{inputenc}
% \usepackage[T1]{fontenc}
% \usepackage{tikz}
% \usepackage{amsmath}
% \usepackage{graphics}
% \usepackage{graphicx}
% \usepackage{url}
% \usepackage{psfrag}
% \usepackage{fancyhdr}
% \usepackage{vmargin}
% \usepackage[backend=biber]{biblatex}
% \usepackage{csquotes}
% \usepackage[hidelinks]{hyperref}
% \usepackage{enumitem}

% \pagestyle{fancy}


% \begin{document}
\subsubsection*{\large{Réunion d'équipe du 23 décembre 2020}}
    \addcontentsline{toc}{subsubsection}{Réunion d'équipe du 23 décembre 2020}
\begin{center}
\begin{tabular}{| l | l || c | c |}
    \hline
    Membres présents & Membres absents & Durée & Lieu \\
    \hline
    Mohamed-Omar CHIDA & & & \\ Mathis DUMAS & & 3h & Discord \\ Chaima TOUNSI OMEZZINE & & & \\ Céline ZHANG & & & \\
    \hline
\end{tabular}
\end{center}

\subsubsection*{Ordre du jour}
\begin{enumerate}
    \item Avancement de chacun sur le projet (fin de la deuxième partie, troisième partie test, quatrième partie discussion)
    \item Résolution des difficultés rencontrées
    \item Optimisation des solutions proposées
    \item Vérification simpliste des tests
    \item Planification du prochain \textsl{sprint}
\end{enumerate}

\subsubsection*{Avancement de chacun sur le projet}
\paragraph*{Omar CHIDA} a continué l'implémentation de l'algorithme de Needleman-Wunsch, la fonction \texttt{needleman} pour la question 7 et la fonction \texttt{needleman\_all} question 9.

\paragraph*{Mathis DUMAS} a testé \texttt{needleman} et a trouvé des problèmes (de \textsl{out of range}) a ajouté les tests de la fonction \texttt{test\_call\_stat\_prop}.

\paragraph*{Chaima TOUNSI OMEZZINE} a appliqué la fonction \texttt{lev\_itr} (la distance de Levenshtein) sur les paires de séquences et a trouvé un problème \textsl{RecursionError} en appliquant \texttt{lev\_dp} d'où la proposition d'une fonction itérative, a structuré la partie gestion de projet dans le rapport, a fait la fonction \texttt{start\_to\_stop}.

\paragraph*{Céline ZHANG} a ajouté des paires de séquences pour l'application de la fonction \texttt{lev\_itr} (calculant la distance de Levenshtein), a appliqué sur celles-ci et a ajouté les tests pour les fonctions \texttt{call\_stats}, \texttt{call\_stats\_taille\_genome}, \texttt{perform\_all\_stats}, \texttt{perform\_all\_stats\_taille}, etc. et a fait la mise en page du rapport.

\subsubsection*{Résolution des difficultés rencontrées}
Il y a eu un problème de \textsl{out of range} sur les fonctions \texttt{needleman}. Le soucis a été réglé par Omar CHIDA.\\
Il y a eu un problème aussi de \textsl{RecursionError} dans l'application de la fonction de \texttt{Levenshtein}. On a réglé le soucis par l'implémentation d'une version \texttt{Levenshtein} itérative.

\subsubsection*{Optimisation des solutions proposées}
Céline ZHANG a fait une fonction \texttt{bank\_sequences} (et une recursive), qui permet de prendre dans une banque de séquences de nucléotides un échantillon donné (inférieur ou égal à 10000).

\subsubsection*{Vérification simpliste des tests}
Les nouveaux tests ajoutés ont été verifiés.

\subsubsection*{Planification du prochain \textsl{sprint}}
\paragraph*{Omar CHIDA} doit finir les tests des fonctions \texttt{needleman}, puis doit commencer les performances des fonctions.

\paragraph*{Mathis DUMAS} doit faire des histogrammes pour les résultats statisques pour la question 2, doit faire les tests pour les fonctions \texttt{needleman}.

\paragraph*{Chaima TOUNSI OMEZZINE} doit corriger la fonction \texttt{codons}, faire les tests de celle-ci, réappliquer sur les séquences de codons.

\paragraph*{Céline ZHANG} doit écrire les résultats pour la première partie avec les applications et les graphes, appliquer sur un échantillon de 10000 séquences, finir la banque de séquences.

\paragraph{\emph{TO-DO LIST}}
\begin{itemize}
    \item Finir les fonctions \texttt{needleman} avec les tests
    \item Faire les graphes pour les premières parties
    \item Corriger la fonction \texttt{codons}
    \item Appliquer sur des échantillons des mois de mars et d'avril
    \item Continuer les conclusions des premières parties et la gestion de projet
\end{itemize}

\emph{Prochaine réunion : 26/12/2020}\\

% \end{document}