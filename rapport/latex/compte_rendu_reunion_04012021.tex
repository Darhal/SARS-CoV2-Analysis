% \documentclass[12pt]{article}
% \usepackage[frenchb]{babel}
% \usepackage{natbib}
% \usepackage[utf8]{inputenc}
% \usepackage[T1]{fontenc}
% \usepackage{tikz}
% \usepackage{amsmath}
% \usepackage{graphics}
% \usepackage{graphicx}
% \usepackage{url}
% \usepackage{psfrag}
% \usepackage{fancyhdr}
% \usepackage{vmargin}
% \usepackage[backend=biber]{biblatex}
% \usepackage{csquotes}
% \usepackage[hidelinks]{hyperref}
% \usepackage{enumitem}

% \pagestyle{fancy}


% \begin{document}
\subsubsection*{\large{Réunion d'équipe du 4 Janvier 2021}}
    \addcontentsline{toc}{subsubsection}{Réunion d'équipe du 4 Janvier 2021}
\begin{center}
\begin{tabular}{| l | l || c | c |}
    \hline
    Membres présents & Membres absents & Durée & Lieu \\
    \hline
    Mohamed-Omar CHIDA & & & \\ Mathis DUMAS & & 4h & Discord \\ Chaima TOUNSI OMEZZINE & & & \\ Céline ZHANG & & & \\
    \hline
\end{tabular}
\end{center}

\subsubsection*{Ordre du jour}
\begin{enumerate}
    \item Avancement de chacun sur le projet (Fin du rapport, début Présentation et discussion sur les démonstrations)
    \item Résolution des difficultés rencontrées
    \item Optimisation des solutions proposées
    \item Vérification simpliste des tests
    \item Planification du prochain \textsl{sprint}
\end{enumerate}

\subsubsection*{Avancement de chacun sur le projet}
\paragraph*{Omar CHIDA} a terminé la fonction permettant l'affichage du traçage pour l'alignement global.

\paragraph*{Mathis DUMAS} a terminé les graphiques pour les fonctions statistiques.

\paragraph*{Chaima TOUNSI OMEZZINE} a ajouté des précisions pour expliquer le tableau de bord Trello et a relu le rapport.

\paragraph*{Céline ZHANG} a rédigé le résumé en français du rapport, corrigé l'introduction en ajoutant le cahier des charges, a réalisé le bilan, et a relu les parties du rapport.

\subsubsection*{Résolution des difficultés rencontrées}


\subsubsection*{Optimisation des solutions proposées}
Tentative d'optimisation des fonctions \texttt{needleman} pour un \textsl{trace back} plus rapide.

\subsubsection*{Vérification simpliste des tests}
Check de l'ensemble des tests écrits.

\subsubsection*{Planification du prochain \textsl{sprint}}
L'ensemble de l'équipe doit relire le rapport, signaler et corriger les fautes. Elle devra réfléchir à la conclusion du rapport, puis à la présentation et démonstration à suivre.

\paragraph{\emph{TO-DO LIST}}
\begin{itemize}
    \item Relecture du rapport
    \item Réfléchir à la conclusion
    \item Réfléchir à la présentation
    \item Réfléchir à la démonstration
\end{itemize}

\emph{Prochaine réunion : 05/01/2021}\\

% \end{document}