% \documentclass[12pt]{article}
% \usepackage[frenchb]{babel}
% \usepackage{natbib}
% \usepackage[utf8]{inputenc}
% \usepackage[T1]{fontenc}
% \usepackage{tikz}
% \usepackage{amsmath}
% \usepackage{graphics}
% \usepackage{graphicx}
% \usepackage{url}
% \usepackage{psfrag}
% \usepackage{fancyhdr}
% \usepackage{vmargin}
% \usepackage[backend=biber]{biblatex}
% \usepackage{csquotes}
% \usepackage[hidelinks]{hyperref}
% \usepackage{enumitem}

% \pagestyle{fancy}


% \begin{document}
\subsubsection*{\large{Réunion d'équipe du 5 décembre 2020}}
    \addcontentsline{toc}{subsubsection}{Réunion d'équipe du 5 décembre 2020}
\begin{center}
\begin{tabular}{| l | l || c | c |}
    \hline
    Membres présents & Membres absents & Durée & Lieu \\
    \hline
    Mohamed-Omar CHIDA & & & \\ Mathis DUMAS & & 2h & Discord \\ Chaima TOUNSI OMEZZINE & & & \\ Céline ZHANG & & & \\
    \hline
\end{tabular}
\end{center}

\subsubsection*{Ordre du jour}
\begin{enumerate}
    \item Avancement de chacun sur le projet (deuxième partie conclusion et troisième partie discussion)
    \item Résolution des difficultés rencontrées
    \item Optimisation des solutions proposées
    \item Vérification simpliste des tests
    \item Planification du prochain \textsl{sprint}
\end{enumerate}

\subsubsection*{Avancement de chacun sur le projet}
Nous avons discuté de l'avancement de chacun sur nos tâches, en partageant ce qui a été réalisé, et nous avons interpréter les résultats des applications de Céline ZHANG, sur les échantillons de séquences de nucléotides. On déduit, sur un échantillon de dix mille séquences entières (du mois de octobre et novembre), l'ARNm du génome SARS-COV2 a une taille d'environ 29813 nucléotides, avec un écart-type de 43 et une intervalle interquartile de 35. Il a environ 8901 nucléotides A, 9580 nucléotides U, 5849 nucléotides G, 5474 nucléotides C, avec respectivement d'écart-type 20 pour A, 20 pour U, 11 pour G, 12 pour C, et d'intervalle interquartile 12 pour A, 17 pour U, 4 pour G, 8 pour C. Des applications sur des échantillons de mille et de deux cents ont été faites, on remarque que les moyennes et médianes ne diffèrent que de très peu (de quelques nucléotides) cependant l'écart-type grandissait avec la taille de l'échantillon, en effet plus la taille de l'échantillon est grande plus on trouve d'écart de valeur. Mais cette augmentation est petite (de l'ordre de 20 nucléotides), ceci est minime par rapport à l'augmentation de la taille de l'échantillon.
\paragraph*{Omar CHIDA} a fixé un problème de \texttt{codons}, a refait les tests de celui-ci, a optimisé les fonctions \texttt{call\_stat\_on\_echantillon}, \texttt{perform\_all\_stats}, s'est proposé d'implémenter la distance de Levenshtein.

\paragraph*{Mathis DUMAS} a écrit sa fonction \texttt{call\_stat\_prop} pour la fonction \texttt{proportions}, a commencé les graphes pour l'illustration de la première et la deuxième partie, et s'est documenté sur la distance de Levenshtein.

\paragraph*{Chaima TOUNSI OMEZZINE} a commencé les tests d'applications des fonctions statistiques sur les séquences d'acides aminés, a continué les tests des fonctions de type \texttt{call\_stat\_on\_echantillon}, et s'est documentée pour la troisième partie, la distance de Levenshtein.

\paragraph*{Céline ZHANG} a appliqué les fonctions statistiques sur des échantillons de séquence de nucléotides, a sauvegardé les résultats pour l'interprétation générale, et s'est documentée pour la troisième partie, la distance de Levenshtein.

\subsubsection*{Résolution des difficultés rencontrées}
Il y a eu des problèmes des conversions de documents, des questions sur le 'Y' qui peut représenter la nucléotide C ou T, donc il y a eu une séance de débat pour finalement envoyer un mail aux encadrants, mais plusieurs solutions possibles ont été pensées. Ceci bloque un peu la partie représentation graphique et l'interprétation des résultats.

\subsubsection*{Optimisation des solutions proposées}
Céline ZHANG a fait des tests pour un échantillon de dix mille séquences de nucléotides prise entre les mois d'octobre et nomvembre, il faut envisager de prendre un échantillon des mois de mars et avril pour comparer les changements éventuels.

\subsubsection*{Vérification simpliste des tests}
Nous avons vu certaines coquille dans les descriptions de fonctions qui ont été corrigées. Nous avons vérifier les tests des fonctions \texttt{proportions} et \texttt{total\_elements}.

\subsubsection*{Planification du prochain \textsl{sprint}}
\paragraph*{Omar CHIDA} doit écrire la fonction qui permet de calculer la distance de Levenshtein entre deux chaînes de caractères.

\paragraph*{Mathis DUMAS} doit commencer l'écriture du rapport avec l'introduction et l'état de l'art, et continuer les graphes.

\paragraph*{Chaima TOUNSI OMEZZINE} doit se documenter et discuter avec Céline ZHANG sur l'implémentation pour la quatrième partie de l'algorithme de Needleman-Wunsch, et réfléchir éventuellement à la conclusion des questions 3, 4.

\paragraph*{Céline ZHANG} doit se documenter et discuter avec Chaima TOUNSI OMEZZINE sur l'implémentation pour la quatrième partie de l'algorithme de Needleman-Wunsch, et commencer éventuellement l'écriture des conclusions des questions 1 et 2.

\paragraph{\emph{TO-DO LIST}}
\begin{itemize}
    \item Implémenter la fonction calculant la distance de Levenshtein
    \item Commencer l'écriture du rapport, introduction et l'état de l'art
    \item Se documenter et réfléchir à l'implémentation de l'algorithme de Needleman-Wunsch
    \item Réfléchir sur la conclusion des applications de la parties 1, 2, 3 et 4.
    \item Envoyer un mail qui permet de lever le doute sur le 'Y', START \textsl{and} STOP, et la notion de \textsl{dynamic programming}
\end{itemize}

\emph{Prochaine réunion : 20/12/2020}\\

% \end{document}