% \documentclass[12pt]{article}
% \usepackage[frenchb]{babel}
% \usepackage{natbib}
% \usepackage[utf8]{inputenc}
% \usepackage[T1]{fontenc}
% \usepackage{tikz}
% \usepackage{amsmath}
% \usepackage{graphics}
% \usepackage{graphicx}
% \usepackage{url}
% \usepackage{psfrag}
% \usepackage{fancyhdr}
% \usepackage{vmargin}
% \usepackage[backend=biber]{biblatex}
% \usepackage{csquotes}
% \usepackage[hidelinks]{hyperref}
% \usepackage{enumitem}

% \pagestyle{fancy}


% \begin{document}
\subsubsection*{\large{Réunion d'équipe du 28 novembre 2020}}
    \addcontentsline{toc}{subsubsection}{Réunion d'équipe du 28 novembre 2020}
\begin{center}
\begin{tabular}{| l | l || c | c |}
    \hline
    Membres présents & Membres absents & Durée & Lieu \\
    \hline
    Mohamed-Omar CHIDA & & & \\ Mathis DUMAS & & 3h & Discord \\ Chaima TOUNSI OMEZZINE & & & \\ Céline ZHANG & & & \\
    \hline
\end{tabular}
\end{center}

\subsubsection*{Ordre du jour}
\begin{enumerate}
    \item Avancement de chacun sur le projet (première partie conclusion et deuxième partie test)
    \item Résolution des difficultés rencontrées
    \item Optimisation des solutions proposées
    \item Vérification simpliste des tests
    \item Planification du prochain \textsl{sprint}
\end{enumerate}

\subsubsection*{Avancement de chacun sur le projet}
Nous avons discuté de l'avancement de chacun sur nos tâches, en partageant ce qui a été réalisé, et en interprétant nos résultats pour la première partie.
\paragraph*{Omar CHIDA} a fait la fonction \texttt{codons} et les écrit des débuts de tests.

\paragraph*{Mathis DUMAS} a fait la fonction \texttt{codons\_echantillon} qui permet l'application sur un échantillon.

\paragraph*{Chaima TOUNSI OMEZZINE} a écrit la fonction \texttt{nombre\_element\_echantillon} et les fonctions d'applications des fonctions statisques sur les séquences d'acides aminés. En remarquant la redondance de code, elle a proposé une optimisation pour englober toutes les fonctions.

\paragraph*{Céline ZHANG} a regroupé les résultats d'application des fonctions statistiques sur les échantillons de séquence (sur plusieurs allant de 10 à 10000), puis a corrigé la fonction \texttt{quartile} et l'a testé (\texttt{intervalle\_interquartile} aussi).

\subsubsection*{Résolution des difficultés rencontrées}
Il y a eu des problèmes d'import des séquences du génome, qui a pu être réglé. On a corrigé certains tests (comme pour \texttt{nombre\_elements}) qui a eu un problème.

\subsubsection*{Optimisation des solutions proposées}
Après des longues discussions sur la façon d'optimistation, Omar a réalisé la fonction qui permet l'application de toutes fonctions statistiques à n'importe quel échantillon donné dans n'importe quelle base (nucléotides ou acides aminés).

\subsubsection*{Vérification simpliste des tests}
Il y a eu de petites corrections sur les tests suite à des modifications. Certains comme les tests de la fonction \texttt{nombre\_elements} ont été refait.

\subsubsection*{Planification du prochain \textsl{sprint}}
\paragraph*{Omar CHIDA} doit finir les tests pour la fonction \texttt{codons}, les fonctions type génératrices et commencer les tests.

\paragraph*{Mathis DUMAS} doit combiner les fonctions statistiques à la fonctions \texttt{proportions}, et doit commencer à réaliser des graphes qui vont permettre d'illuster les résultats de la première partie.

\paragraph*{Chaima TOUNSI OMEZZINE} doit tester ses fonctions d'application sur les acides aminés pour la deuxième partie, et regarder la troisième partie pour la distance de Levenshtein.

\paragraph*{Céline ZHANG} doit réfléchir à la rédaction de la conlusion de la première partie avec les résultats, finir les tests inachevés, et regarder la troisième partie pour la distance de Levenshtein.

\paragraph{\emph{TO-DO LIST}}
\begin{itemize}
    \item Continuer et finir les fonctions type génératrice
    \item Réfléchir à la conclusion de la première et deuxième partie
    \item Commencer à écrire les fonctions qui affichent les graphes des résultats statistiques
\end{itemize}

\emph{Prochaine réunion : 05/12/2020}\\

% \end{document}