% \documentclass[12pt]{article}
% \usepackage[frenchb]{babel}
% \usepackage{natbib}
% \usepackage[utf8]{inputenc}
% \usepackage[T1]{fontenc}
% \usepackage{tikz}
% \usepackage{amsmath}
% \usepackage{graphics}
% \usepackage{graphicx}
% \usepackage{url}
% \usepackage{psfrag}
% \usepackage{fancyhdr}
% \usepackage{vmargin}
% \usepackage[backend=biber]{biblatex}
% \usepackage{csquotes}
% \usepackage[hidelinks]{hyperref}
% \usepackage{enumitem}

% \pagestyle{fancy}


% \begin{document}
\subsubsection*{\large{Réunion d'équipe du 21 novembre 2020}}
    \addcontentsline{toc}{subsubsection}{Réunion d'équipe du 21 novembre 2020}
\begin{center}
\begin{tabular}{| l | l || c | c |}
    \hline
    Membres présents & Membres absents & Durée & Lieu \\
    \hline
    Mohamed-Omar CHIDA & & & \\ Mathis DUMAS & & 3h & Discord \\ Chaima TOUNSI OMEZZINE & & & \\ Céline ZHANG & & & \\
    \hline
\end{tabular}
\end{center}

\subsubsection*{Ordre du jour}
\begin{enumerate}
    \item Avancement de chacun sur le projet (première partie test et deuxième partie)
    \item Résolution des difficultés rencontrées
    \item Optimisation des solutions proposées
    \item Vérification simpliste des tests
    \item Planification du prochain \textsl{sprint}
\end{enumerate}

\subsubsection*{Avancement de chacun sur le projet}
Nous avons discuté de l'avancement de chacun sur nos tâches, en partageant ce qui a été réalisé.
\paragraph*{Omar CHIDA} a testé \texttt{fasta\_to\_genome} nettoyé un peu les fichiers inutiles, a corrigé le style d'écriture des fonctions (commentaires, tabulations, saut de ligne, etc.), a résolu les conflicts.

\paragraph*{Mathis DUMAS} a testé sa fonction \texttt{transcription\_complementaire}, a rendu plus souple le code de certaines fonctions (\texttt{fasta\_to\_genome}, \texttt{nombre\_nucleotide}, etc.) et le dictionnaire des acides aminés.

\paragraph*{Chaima TOUNSI OMEZZINE} a testé ses fonctions \texttt{nombre\_nucleotide} et \texttt{total\_nucleotide} et a apporté des optimisations.

\paragraph*{Céline ZHANG} a testé ses fonctions stastiques de \texttt{stats.py} et ses fonctions de \texttt{utility.py}, a ajouté les fonctions d'application des fonctions d'analyse statistique sur les échantillons de séquences (\texttt{moyenne\_nucleotides}, \texttt{quartile\_nucleotides}, \texttt{variance\_nucleotides}, \texttt{taille\_genome}, \texttt{moyenne\_taille\_genome}, etc.) qui permettent de décrire le génome du SARS-COV2.

\subsubsection*{Résolution des difficultés rencontrées}
Céline a un doute avec la fonction quartile après un débat, nous avons préféré demander à Madame MÉZIÈRES, pour lever le doute sur le calcul de la fonction quartile (car il y avait plusieurs interpolation possible). Plusieurs diverses problèmes (coquilles, structures, etc.) ont été réglé.

\subsubsection*{Optimisation des solutions proposées}
On a proposé une fonction qui regroupe l'application des fonctions statistiques car l'appel de ces fonctions est assez répétitif. Omar a noté cette idée qui se trouvera dans le prochain \textsl{sprint}. On a regroupé les fonctions pour nucléotides et acides aminés en un type de fonction ayant pour terminaison \texttt{\_elements}.

\subsubsection*{Vérification simpliste des tests}
Les tests proposés ont bien fonctionné et ont été approuvés.

\subsubsection*{Planification du prochain \textsl{sprint}}
\paragraph*{Omar CHIDA} doit réfléchir à la partie concernant les condons et réaliser la fonction \texttt{codons}.

\paragraph*{Mathis DUMAS} doit réfléchir à la partie concernant les condons et implémenter le nécessaire pour les applications de \texttt{codons}.

\paragraph*{Chaima TOUNSI OMEZZINE} doit regarder la partie pour les acides aminés, faire les applications.

\paragraph*{Céline ZHANG} résoudre le problème de quartile, corriger la fonction \texttt{quartile} et finir les tests des fonctions \texttt{quartile}, \texttt{intervalle\_interquartile}.

\paragraph{\emph{TO-DO LIST}}
\begin{itemize}
    \item Implémenter la fonction \texttt{codons}
    \item Appliquer les fonctions statisques sur les acides aminés
    \item Résoudre le problème de la fonction \texttt{quartile} et refaire les tests pour les fonctions \texttt{quartile} et \texttt{intervalle\_interquartile}
\end{itemize}

\emph{Prochaine réunion : 28/11/2020}\\

% \end{document}