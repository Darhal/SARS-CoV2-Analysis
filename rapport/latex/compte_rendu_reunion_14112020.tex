% \documentclass[12pt]{article}
% \usepackage[frenchb]{babel}
% \usepackage{natbib}
% \usepackage[utf8]{inputenc}
% \usepackage[T1]{fontenc}
% \usepackage{tikz}
% \usepackage{amsmath}
% \usepackage{graphics}
% \usepackage{graphicx}
% \usepackage{url}
% \usepackage{psfrag}
% \usepackage{fancyhdr}
% \usepackage{vmargin}
% \usepackage[backend=biber]{biblatex}
% \usepackage{csquotes}
% \usepackage[hidelinks]{hyperref}
% \usepackage{enumitem}

% \pagestyle{fancy}


% \begin{document}
\subsubsection*{\large{Réunion d'équipe du 14 novembre 2020}}
    \addcontentsline{toc}{subsubsection}{Réunion d'équipe du 14 novembre 2020}
\begin{center}
\begin{tabular}{| l | l || c | c |}
    \hline
    Membres présents & Membres absents & Durée & Lieu \\
    \hline
    Mohamed-Omar CHIDA & & & \\ Mathis DUMAS & & 5h & Discord  \\ Chaima TOUNSI OMEZZINE & & & \\ Céline ZHANG & & &\\
    \hline
\end{tabular}
\end{center}

\subsubsection*{Ordre du jour}
\begin{enumerate}
    \item Avancement de chacun sur le projet (première partie)
    \item Résolution des difficultés rencontrées
    \item Optimisation des solutions proposées
    \item Vérification simpliste des tests
    \item Planification du \textsl{sprint}
\end{enumerate}

\subsubsection*{Avancement de chacun sur le projet}
Nous avons discuté de l'avancement de chacun sur la première partie du sujet, en mettant en commun nos idées, nos découvertes et nos codes.
\paragraph*{Omar CHIDA} a organisé le tableau de bord Trello~\ref{fig:Trellodebut}, a mis une structure pour accéder au fichier par import pour les tests, il a proposé la première version de la fonction \texttt{fasta\_to\_genome} qui se trouve dans \texttt{utility.py}. %, très utile pour exploiter les fichiers \texttt{.fasta} dans nos tests.

\paragraph*{Mathis DUMAS} a écrit plusieurs premières versions des fonctions de \texttt{utility.py} et de \texttt{stats.py} (comme \texttt{proportion}, \texttt{transcription\_complementaire}).

\paragraph*{Chaima TOUNSI OMEZZINE} a proposé les versions primaires des fonctions \texttt{total\_nucleotide}, \texttt{nombre\_nucleotide} dans \texttt{utility.py}

\paragraph*{Céline ZHANG} a codé plusieurs fonctions statistiques se trouvant dans \texttt{stats.py} et \texttt{utility.py} (comme les fonctions \texttt{taille\_ensemble}, \texttt{moyenne}, \texttt{mediane}, \texttt{quartile}, \texttt{variance}, etc.)

\subsubsection*{Résolution des difficultés rencontrées}
Il y a eu des problèmes diverses rencontrés, la résolution s'est fait en équipe. Céline ZHANG avait remarqué un problème d'import des fichiers dans lesquels se trouvait les codes à tester, Omar CHIDA a réglé le problème. Mathis DUMAS a avait quelques soucis sur ses fonctions qu'il a pu debug. Chaima TOUNSI avait des conflicts lors des pull, push, merge, elle a pu réglé ça avec Omar.

\subsubsection*{Optimisation des solutions proposées}
Dans nos fonctions, les membres ont constaté qu'il pouvait y avoir des améliorations en celui-ci, comme les ajouts de fonctionnalité, faire dans \texttt{fasta\_to\_genome} le changement direct du T en U, les améliorations du dictionnaire, et les optimisations de \texttt{transcription\_complementaire} en modifiant l'itération. Tous les membres ont participés à la réflexion et proposition de solution.

\subsubsection*{Vérification simpliste des tests}
Parmis les tests déjà réalisés, nous avons vérifié la validité pour tout le monde, puis nous avons réalisé des tests simples pour les fonctions (\texttt{fasta\_to\_genome}, \texttt{transcription\_complementaire}, etc.) que nous avons trouvé urgent de tester à ce moment là pour la suite du parcours (bien sûr elles vont être retesté formellement).

\subsubsection*{Planification du \textsl{sprint}}
Enfin, la réunion se termine par la planification de notre premier vrai \textsl{sprint}. On sélectionne les tâches à faire pour le cycle (une semaine), et on pré-répartie quelques tâches par préférence ou compétence des membres, et le reste est à prendre dès qu'on finit nos tâches (l'asignement se fait sur Trello).
\paragraph*{Omar CHIDA} doit continuer dans les fonctions basiques, avance sur la documentation des autres parties.

\paragraph*{Mathis DUMAS} doit tester ses fonctions et réfléchir l'affichage des tests d'échantillon.

\paragraph*{Chaima TOUNSI OMEZZINE} doit tester ses fonctions et réfléchir sur la partie des acides aminés.

\paragraph*{Céline ZHANG} doit tester ses fonctions, continuer sur les fonctions de statistiques descriptives, et réfléchir à comment appliquer sur le génome.
% mettre annexe un screen du trello ou autre qui illustre le sprint

\paragraph{\emph{TO-DO LIST}}
\begin{itemize}
    \item Continuer et finir les fonctions statistiques (interquartile, etc.) de \texttt{stats.py}
    \item Continuer les fonctions basiques de \texttt{utility.py} (pour nucléotides, acides, échantillon)
    \item Faire et écrire les tests des fonctions déjà réalisée (le plus possible)
\end{itemize}

\emph{Prochaine réunion : 21/11/2020}\\

% \end{document}